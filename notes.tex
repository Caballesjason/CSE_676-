% ----- META DATA ----
% Declare document type
\documentclass[12pt]{article}
% Set paper marign
\usepackage[letterpaper, portrait, margin=0.5in]{geometry}
% Set font to times new roman
\usepackage{fontspec}
\setmainfont{Times New Roman}
% set font for section titles
\usepackage{sectsty}
\sectionfont{\fontsize{16}{12}\selectfont}
\subsectionfont{\fontsize{14}{12}\selectfont}
\subsubsectionfont{\fontsize{14}{12}\selectfont}
% Adjust paragraph spacing and indentation
\usepackage[skip=6pt, indent=0pt]{parskip}
% Set spacing between sections
\usepackage{setspace}
\singlespacing
% Used to add image to title screen
\usepackage{graphicx}

% Set path to images directory
\graphicspath{{images/}} 
% Used for multiple line line comments
\usepackage{comment}
% Used to reduce spacing between text and section headers
\usepackage{titlesec}
\titlespacing\section{0pt}{6pt plus 4pt minus 2pt}{0pt plus 2pt minus 2pt}
\titlespacing\subsection{0pt}{6pt plus 4pt minus 2pt}{0pt plus 2pt minus 2pt}
\titlespacing\subsubsection{0pt}{6pt plus 4pt minus 2pt}{0pt plus 2pt minus 2pt}
% Used for table of contents
\usepackage{blindtext}
% Used for writing text within math equations
\usepackage{amsmath}
% Used for adding multiline cells in tables
\usepackage{makecell}
% Used to write algorithms
\usepackage{algpseudocode}
% Used to write python code
\usepackage{pythonhighlight}

% ---- TITLE PAGE ----
\title{
  CSE 676 - Introduction to Deep Learning
}
\author{
  Jason Caballes
}

% ---- START OF DOCUMENT ----
\begin{document}
% Create title page
% \maketitle
% \newpage

% Adding page numbers
\pagenumbering{arabic}


% ---- TABLE OF CONTENT ----
\begin{comment}
\tableofcontents
\newpage
\end{comment}

% ---- SLIDE DECK 1 ----
% ---- COURSE OVERVIEW ----
\section*{Course Overview}
\subsubsection*{Course Logistics}
\subsubsection*{Course Outline}
\subsubsection*{Grading Rubric}
\subsubsection*{Machine Learning Review}
\subsubsection*{Artificial Intelligence}

% ---- ARTIFICIAL INTELLIGENCE OVERVIEW ----
\section{Artifical Intelligence Overview}
\subsubsection{AI Paradox}
Artifical Narrow Intelligence (ANI) vs Artificial General Intelligence (AGI)

\begin{itemize}
  \item Problems difficult for humans are easy for AI
  \item Problems easy for humans are difficult for AI
\end{itemize}

\subsection{Knowledge-Based AI}
\begin{itemize}
  \item AI that is provided explicity rules
  \item People struggle to formalize these rules
\end{itemize}

\subsubsection{Machine Learing Approach}
\begin{itemize}
  \item Allows computers to learn from experience
  \item Learns to map features to outputs
\end{itemize}

\subsubsection{Supervised vs Unsupervised Learning}

\begin{itemize}
  \item Dataset: Collection of unlabeled examples $\{x_{i}\}^{N}_{i=1}$
  \item Goal:  Create a model tha ttakes $x$ as input and either transform it into another vector or into a value that can be used to solve a practical problem
\end{itemize}

\textbf{Note}:  Training a pretrained model for with a training and test set is considered semi-supervised learning

\subsubsection{Reinforcement Learning}
\textbf{Input}: State-action pairs \\
\textbf{Goal}: Learn a good sequence of decisions to maximize a reward

\section{Preliminaries}
\subsubsection{Data Manipulation}
\begin{itemize}
  \item A tensor represents a (possibly multi-dimensional) array of numerical values
  \begin{itemize}
    \item With one axis, a tensor is called a \textit{vector}
    \item With two axes, a tensor is called a \textit{matrix}
    \item With $n > 2$ axes, we just call it a tensor
  \end{itemize}
\end{itemize}

\begin{python}
# Library import
import torch
# Assign x to an array with 12 floats
x = torch.arange([12, dtype=torch.float])
\end{python}

\subsubsection{Linear Algebra}
\subsubsection{Calculus}
\subsubsection{Automatic Differentiation}
\subsubsection{Probability and Statistics}


% ---- Linear Neural Networks for Regression ----
\section{Linear Neural Networks for Regression}

% ---- SLIDE DECK 2 ----
% ---- Multiple Layer Perceptron (MLP) ----
\section{Multiple Layer Perceptron (MLP)}

You must have a general understanding of the following ML algorithms

\begin{itemize}
  \item Linear Regression
  \item KNN
  \item K-means Clustering
  \item Linear SVM (\textbf{Important for Interviews})
  \item SVM with gaussian kernel for non-linear SVM
  \item Naive Bayes Classifier
  \item Principal Component Analysis (PCA)
\end{itemize}

You must understand the following performance metrics

\begin{itemize}
  \item ROC
  \item AUC
  \item F1 Score
  \item Precision
  \item Recall
  \item Accuracy
\end{itemize}

% ----  ----
\section{}

\begin{comment}
% ---- REFERENCES ----
All references are in MLA 9TH Gen
\newpage
\begin{thebibliography}{widest entry}
    % \bibitem{O’Halloran} O’Halloran, Sharyn, et al. “Data Science and Political Economy: Application to Financial Regulatory Structure.” RSF : Russell Sage Foundation Journal of the Social Sciences, vol. 2, no. 7, 2016, pp. 87–109, https://doi.org/10.7758/rsf.2016.2.7.06.
    \end{thebibliography}
\end{comment}

% ---- END OF DOCUMENT ----
\end{document}

% Must figure out how to get the max time and date
% https://data.cityofnewyork.us/resource/i4gi-tjb9.json?$query=SELECT speed, travel_time, link_name, borough, encoded_poly_line, data_as_of WHERE status='0' ORDER BY :id LIMIT 100